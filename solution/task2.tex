\documentclass{article}
\usepackage[T2A]{fontenc}
\usepackage[utf8]{inputenc}
\usepackage[russian]{babel}
\begin{document}
\section{Решение квадратного уравнения}

\textit{Задача:} решить уравнение $2x^2+5x-12=0$

\hspace*{5mm}\textit{Решение.} Это квадратное уравнение, общий вид которого:

\hspace*{30mm}$ax^2+bx+c=0$

В нашем случае $a=2$, $b=5$, $c=-12$.

\hspace*{5mm}Сначала необходимо вычислить дискриминант уравнения:

\hspace*{30mm}$D=b^2-4ac=(5)2-4\cdot2\cdot(-12)=121$

Так как дискриминант является положительным ($D>0$), это уравнение
имеет два корня, вычисляемые по формуле:


\hspace*{30mm}$x_{1,2}=\frac{-b\pm\sqrt{D}}{2a}=\frac{-5\pm11}{4}$


Таким образом, $x_{1}=\frac{-16}{4}=-4$, $x_{2}=\frac{6}{4}=\frac{3}{2}$.


\hspace*{5mm}\textit{Ответ:} $x_{1}=-4$, $x_{2}=\frac{3}{2}$.
\end{document}